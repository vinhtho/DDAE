% First draft - Phi 24. Nov 14

\documentclass[final,reqno]{siamltex}

\usepackage{amsfonts,epsfig}
\usepackage{amsmath,amssymb} %$#
\usepackage{graphicx}

\usepackage[square, numbers, comma, sort&compress]{natbib}  % Use the "Natbib" style for the references in the Bibliography
%\usepackage{showlabels}
%\renewcommand{\showlabelfont}{\small\slshape\color{red}}

\usepackage{algorithm}
\usepackage{algpseudocode}
\usepackage{algcompatible}

\usepackage{xcolor}
\newcommand\command[1]{\textcolor{red}{\textbf{#1}}}

% %\usepackage[notcite]{showkeys}
% \usepackage{showlabels}
% \renewcommand{\showlabelfont}{\small\slshape\color{red}}

%\usepackage{lineno}
%\linenumbers

\usepackage{paralist}
\renewenvironment{itemize}[1]{\begin{compactitem}#1}{\end{compactitem}}
\renewenvironment{enumerate}[1]{\begin{compactenum}#1}{\end{compactenum}}
\renewenvironment{description}[0]{\begin{compactdesc}}{\end{compactdesc}}

\usepackage[pdfpagelabels]{hyperref} %$#

% Common extra environments
%\newtheorem{algorithm}[theorem]{Algorithm}
\newtheorem{remark}[theorem]{Remark}
\newtheorem{example}[theorem]{Example}
\newtheorem{ass}[theorem]{Assumption}
\newtheorem{hyp}[theorem]{Hypothesis}
\newtheorem{pro}[theorem]{Procedure}

%=============================================================================
% new def-s and commands
\newcommand{\reals}{\mathbb{R}}
\newcommand{\complex}{\mathbb{C}}

\def\leq{\leqslant}
\def\rar{\rightarrow}

\def\hro{\mathbb}
\def\R{\hro{R}}
\def\C{\hro{C}}
\def\N{\hro{N}}
\def\Z{\hro{Z}}
\def\bbI{\mathbb{I}}

\def\a{\alpha}
\def\ta{\widetilde{\alpha}}
\def\b{\beta}
\def\de{\delta}
\def\De{\Delta}
\def\tet{\theta}
\def\Tet{\Theta}
\def\dch{\dot{\chi}}
\def\ch{\chi}
\def\ka{\kappa}
\def\vp{\varphi}

\def\dtau{\Delta_{\tau}}
\def\idtau{\Delta_{-\tau}}

\def\ga{\gamma}
\def\Si{\Sigma}
\def\si{\sigma}
\def\om{\omega}
\def\lb{\lambda}
\def\fr{\frac}

\def\tp{\tilde{p}}
\def\tu{\tilde{u}}

\def\frE{\mathfrak{E}}
\def\frA{\mathfrak{A}}
\def\frB{\mathfrak{B}}
\def\frC{\mathfrak{C}}
\def\frD{\mathfrak{D}}
\def\frg{\mathfrak{g}}
\def\frf{\mathfrak{f}}
\def\frX{\mathfrak{X}}
\def\frZ{\mathcal{Z}}

\def\frM{\mathcal{M}}
\def\frP{\mathcal{P}}
\def\frG{\mathfrak{G}}
\def\tfrM{\tilde{\mathcal{M}}}
\def\tfrN{\tilde{\mathcal{N}}}
\def\tfrP{\tilde{\mathcal{P}}}
\def\tfrG{\tilde{\mathcal{G}}}

\def\hfrg{\hat{\mathfrak{g}}}
\def\tfrg{\tilde{\mathfrak{g}}}
\def\hmu{\hat{\mu}}

\def\bbM{\mathbb{M}}
\def\bbN{\mathbb{N}}

\def\hr{\hat{r}}
\def\hv{\hat{v}}
\def\hm{\hat{m}}
\def\hw{\hat{w}}

\def\tP{\tilde{P}}
\def\tQ{\tilde{Q}}
\def\tH{\tilde{H}}
\def\tK{\tilde{K}}

\def\tE{\widetilde{E}}
\def\hE{\hat{E}}
\def\tA{\widetilde{A}}
\def\hA{\hat{A}}
\def\bA{\breve{A}}
\def\tB{\widetilde{B}}
\def\hB{\hat{B}}

\def\hR{\hat{R}}
\def\hS{\hat{S}}
\def\hT{\hat{T}}

\def\tR{\tilde{R}}
\def\tS{\tilde{S}}
\def\tT{\tilde{T}}

\def\tU{\tilde{U}}
\def\tk{\tilde{k}}
\def\hk{\hat{k}}
\def\tX{\tilde{X}}
\def\hX{\hat{X}}
\def\bX{\breve{X}}

\def\tm{\tilde{m}}
\def\bM{\breve{M}}
\def\hM{\hat{M}}
\def\tM{\widetilde{M}}

\def\tmu{\tilde{\mu}}

\def\bE{\breve{E}}
\def\bA{\breve{A}}
\def\bB{\breve{B}}
%\def\uE{\u{E}}
%\def\uA{\u{A}}
%\def\uB{\u{B}}

\def\bbM{\mathbb{M}}
\def\btM{\widetilde{\mathbb{M}}}

\def\hW{\widehat{W}}
\def\tW{\tilde{W}}

\def\hN{\widehat{N}}
\def\tN{\widetilde{N}}

\def\cP{{\cal P}}
\def\cQ{{\cal Q}}
\def\cU{{\cal U}}

\def\cF{{\cal F}}
\def\cG{{\cal G}}
\def\hcF{\hat{\cF}}
\def\hcG{\hat{\cG}}

\def\hcP{\hat{\cP}}
\def\hcQ{\hat{\cQ}}



\def\hS{\widehat{S}}
\def\hZ{\widehat{Z}}
\def\hH{\hat{H}}
\def\hG{\hat{G}}

\def\tx{\tilde{x}}
\def\tf{\tilde{f}}
\def\hf{\hat{f}}
\def\brf{\breve{f}}
\def\brg{\breve{g}}
\def\baf{\bar{f}}
\def\tg{\tilde{g}}
\def\hg{\hat{g}}
\def\ha{\hat{a}}
\def\hd{\hat{d}}
\def\hu{\hat{u}}

\def\cg{\mathcal{g}}
\def\ti{\times}
\def\utau{\underline{\tau}}
\def\otau{\bar{\tau}}
\def\vtau{{\tau}}
%\def\vtau{\vec{\tau}}
\def\hga{\hat{\ga}}

\renewcommand{\th}[1]{t_{(#1)}}

\def\be{\begin{equation}}
\def\ee{\end{equation}}
\newcommand{\ben}{\begin{eqnarray}}
\newcommand{\een}{\end{eqnarray}}
\newcommand{\bens}{\begin{eqnarray*}}
\newcommand{\eens}{\end{eqnarray*}}
\def\bc{\begin{cases}}
\def\ec{\end{cases}}
\newcommand{\bsq}{\begin{subequations}}
\newcommand{\esq}{\end{subequations}}

\newcommand{\m}[1]{
\begin{bmatrix}
 #1
\end{bmatrix}
}

\renewcommand{\pm}[1]{
\begin{matrix}
 #1
\end{matrix}
}

\newcommand{\doublehat}[1]{%
    \widehat{\widehat{#1\,}}
    }

\newcommand {\mpar}[1]{\marginpar{\fussy\tiny #1}} % marginal notes
\newcommand{\tcal}[1]{%
    \widetilde{\cal{#1}}
    }

    \newcommand{\hcal}[1]{%
    \widehat{\cal{#1}}
    }

    \newcommand{\bcal}[1]{%
    \breve{\cal{#1}}
    }

\newcommand{\bsp}[1]{
\begin{split}
 #1
\end{split}
}
\def\ddt{\fr{\mathrm{d}}{\mathrm{d}t}}

\newcommand{\eproof}{\space
    {\ \vbox{\hrule\hbox{\vrule height1.3ex\hskip0.8ex\vrule}\hrule}}\\[0.2cm]}
    
\def\bce{\begin{compactenum}}
\def\ece{\end{compactenum}}

\newcommand {\corank}   {\mathop{\rm corank}\nolimits}
\newcommand {\range}  {\mathop{\rm range}\nolimits}
\newcommand {\corange}  {\mathop{\rm corange}\nolimits}
\newcommand {\kernel}   {\mathop{\rm kernel}\nolimits}
\newcommand {\cokernel} {\mathop{\rm cokernel}\nolimits}
%===============================================================================
%opening
\begin{document}

\title{A new solver for linear delay differential-algebraic equations\footnotemark[1]}

\author{Phi Ha\footnotemark[2] \and Vinh Tho Ma\footnotemark[2] and Volker Mehrmann\footnotemark[2]}

\renewcommand{\thefootnote}{\fnsymbol{footnote}}

\footnotetext[1]{This work was supported by DFG Collaborative Research Centre 910,
{\it Control of self-organizing nonlinear systems: Theoretical methods and concepts of application}}
\footnotetext[2]{ Institut f\"{u}r Mathematik, MA 4-5, TU Berlin, Stra\ss e des 17. Juni 136, D-10623 Berlin,
Germany; \{ha,vinhma,mehrmann\}@math.tu-berlin.de}

\maketitle

\newcommand{\thedate}{Version 0.2 \quad \today}

\begin{center}
\thedate
\end{center}

\vskip 0.2cm

\begin{abstract}
We discuss the new solver COLDDAE for the numerical solution of initial value problems for linear differential-algebraic equations(DDAEs) with variable coefficients. The implementation is mainly 
based on the results introduced in \cite{HaM14}. The solver can deal with DDAEs in the full general situation, where the system can be either causal or noncausal. 
The hidden type of the system (retarded, neutral, advanced) are determined within the solver, which can deal with not only retarded and neutral systems but also a 
certain class of advanced DDAEs.
\end{abstract}

\begin{keywords} Delay differential-algebraic equation, differential-algebraic equation, delay differential equations, method of steps, derivative array, classification of DDAEs.
\end{keywords}

\begin{AMS}
34A09, 34A12, 65L05, 65H10.
\end{AMS}

\pagestyle{myheadings}
\thispagestyle{plain}
\markboth{P. Ha and V. T. Ma and V. Mehrmann}{COLDDAE: a new solver for linear DDAEs}

\section{Introduction}
We discuss the new solver COLDDAE for the numerical solution of linear delay differential-algebraic equations (DDAEs) with variable coefficients of the following form
%
\bsq\label{eq1.1}
\be\label{eq1.1a}
 E(t)\dot{x}(t) = A(t)x(t) + B(t) x(t-\vtau(t)) + f(t), \quad t\in \mathbb{I} :=[t_0,t_f],
\ee
%
where $E$, $A \in C(\mathbb{I},\mathbb{R}^{m\times n})$, $B = \m{B_1 & \dots & B_k} \in C(\mathbb{I},\mathbb{R}^{m\times kn})$, 
$f\in C(\mathbb{I},\mathbb{R}^{m})$, $\vtau(t) = \m{\tau_1(t) & \dots & \tau_k(t)}$, 
where the delay functions $\tau_i$, $i=1,\dots,k$ satisfy $t > \tau_i(t) > 0$ for all $t\in [t_0,t_f]$.	

Set $\utau := \min \{\tau_i(t)| \ t\in [t_0,t_f], \ i=1,\dots,k \}$ and $\otau := \max \{\tau_i(t)| \ t\in [t_0,t_f], \ i=1,\dots,k \}$, 
we assume that $\utau >0$, which is often referred in the literature \cite{BelC63,BakPT02} as the \emph{non vanishing delay} case. 
We further assume that $\otau < \infty$. Typically, to form an IVP for the DDAE \eqref{eq1.1a}, one needs to add a history function (an initial function) 
%
\be
 x|_{[t_0-\otau,t_0]} = \phi \in C([t_0-\otau,t_0],\R^n).
\ee
%
We assume that the IVP \eqref{eq1.1} has a unique solution $x\in C(\mathbb{I},\mathbb{R}^n)$.\\
\esq
%
The theoretical analysis of IVPs of the form \eqref{eq1.1} has been discussed in \cite{HaMS14,HaM14}. 
We, however, will recall only the appropriate parts of these works to make the procedure of computing the solution of \eqref{eq1.1} transparent.
The most important concepts presented in these works are the causality, the system type, the shift index and the strangeness index of the DDAE \eqref{eq1.1a}.

Under the assumption on the existence and uniqueness of the analytical solution of the IVP \eqref{eq1.1}, the implementation of the new solver is based on the construction 
of the regularization procedure introduced in \cite{HaM14}, which first determines the shift index, strangeness index and then transforms the DDAE \eqref{eq1.1a} into 
the regular, strangeness-free formulation with the same solution set. Using this regularization procedure, we can also compute a consistent initial function and apply the 
well-known integration schemes for the resulting regular, strangeness-free DDAEs. 
Having seen the advantages of collocation Runge-Kutta methods for DDAEs, see \cite{GugH01,GugH07}, in our solver we have implemented collocation Runge-Kutta (RK) schemes. 
% and backward differentiation formula (BDF) methods.
%
%=======================================================================================================================================
\section{A brief survey of the basic results}
%=======================================================================================================================================
%
The numerical solution of IVPs for DDAEs, until now, has only been considered for square systems, see e.g. 
\cite{AscP95,BakPT02,CamL09,GugH07,Liu99,ShaG06,TiaYK11,ZhuP97,ZhuP98}. 
For such systems, the solution is usually computed by the classical (Bellman) \emph{method of steps}, which will be recalled below.\\
Since $\utau>0$, we have $[t_0,t_f] \subset \underset{j=1,\dots,\ell+1}{\cup} [t_0+(j-1)\utau,t_0+j\utau]$ 
with $\ell := \lfloor \fr{t_f-t_0}{\utau} \rfloor$.
For all $t \in [t_0,t_0+\utau]$, we have $t-\tau_i(t) \leq t_0 + \utau - \utau = t_0,$ and hence $x(t-\tau_i(t)) = \phi(t-\tau_i(t))$.
The DDAE \eqref{eq1.1a} restricted on the interval $[t_0,t_0+\utau]$ then becomes
%
\be\label{eq2}
 E(t)\dot{x}(t) = A(t)x(t) + B(t) \phi(t-\vtau(t)) + f(t),
\ee
%
where $\phi(t-\vtau(t)) := \m{\phi^T(t-\tau_1(t)) & \dots & \phi^T(t-\tau_k(t))}^T$. 
This system is the DAE in the variable $x_1 := x|_{[t_0,t_0+\utau]}$. The initial vector of the corresponding IVP for the DAE \eqref{eq2} is $x(t_0)=\phi(0)$.
Suppose that this IVP has a unique solution $x_1$, we can proceed in the same way to compute the function $x_2 := x|_{[t_0+\utau,t_0+2\utau]}$, since 
$t-\tau_i(t) \leq t_0 + 2 \utau - \utau = t_0+\utau$, for all $t\in [t_0+\utau,t_0+2\utau]$. Therefore, the solution $x$ of the IVP \eqref{eq1.1} will be 
computed steps by steps on consecutive intervals $[t_0+(j-1)\utau,t_0+j\utau]$, $1\leq j\leq \ell$. 
On the interval $[t_0+(j-1)\utau,t_0+j\utau]$, the function $x_j := x|_{[t_0+(j-1)\utau,t_0+j\utau]}$ is computed from the DAE of the form
%
\be\label{eq3}
 E(t) \dot{x}_j(t) = A(t)x_j(t) + g_j(t), \quad \mbox{for all } t \in [t_0+(j-1)\utau,t_0+j\utau].
\ee
%
Clearly, we see that the method of steps successfully handle the IVP \eqref{eq1.1} if and only if for every $j$, the corresponding IVP for the 
DAE \eqref{eq3} has a unique solution.
This requirement means that the solution $x$ at a current point $t$ depends only on the system \eqref{eq1.1a} 
at current and past time points (i.e., $s \leq t$), but not future time points ($s > t$). We call this property \emph{causality}, and 
a DDAE that satisfies this property \emph{causal}. 
Restricted to the class of causal systems, different integration strategies based on the method of steps have been successfully implemented for 
linear DDAEs of the form \eqref{eq1.1a} and also for several classes of nonlinear DDAEs, see e.g. \cite{AscP95,BakPT02,GugH07,Hau97,ShaG06}.
In contrast to causal DDAEs, this approach is not feasible for noncausal systems,
consider for example equation 
%
\be\label{eq2.1}
  0 \cdot \dot{x}(t) = 0 \cdot x(t) - x(t-\tau) + f(t), \quad \mbox{for all } t\in (0,\infty),
\ee
%
The method of steps applied to the DDAE \eqref{eq2.1} results in a sequence of undetermined DAEs of the form
%
\[
 0 = g_i(t), \quad \mbox{for all } t \in [t_0+(j-1)\utau,t_0+j\utau]. 
\]
%
Nevertheless, the IVP \eqref{eq1.1} still has a unique solution. 
The reason for this failure is that the method of steps takes into account only the equation at the current time, which is not enough, 
due to the noncausality of general DDAEs. Therefore, a regularization procedure for DDAEs, so that the method of steps can be used for the resulting systems, is 
necessary. Note that for noncausal DDAEs of the form \eqref{eq1.1a}, the solvability analysis has only been discussed for 
the single delay case, i.e., $k=1$. Even for multiple constant delays, i.e., $\tau_i(t) \equiv \tau_i$, the solvability analysis for noncausal DDAEs 
is not entirely understood, \cite{HaM14,Ha15}. 
The regularization procedure proposed in the package COLDDAE for causal (reps. noncausal) systems will be consider in Subsection \ref{Sec2.1} (resp. \ref{Sec2.2}) below.

\subsection{Regularization procedure for causal DDAEs with multiple delays}\label{Sec2.1}
Inherited from the theory of DAEs, we see that even for causal DDAEs, the numerical integration requires a reformulation of the original system in such a way that one can 
avoid the loss in order of convergence or the drift-off effect, see e.g. \cite{BreCP96,KunM06}. Here we use the regularization procedure 
associated with the \emph{strangeness index} concept, see \cite{KunM06}, which generalizes the well-known \emph{differentiation index} \cite{BreCP96} for general 
under- and over-determined DAEs. Briefly speaking, the strangeness index $\mu$ of the DAE 
%
\be\label{eq4}
 E(t)\dot{x}(t) = A(t)x(t) + f(t), 
\ee
%
is the minimum number of differentiation such that from the \emph{derivative arrays} (or \emph{differentiation-inflated} system)
%
\bens
 E(t)\dot{x}(t) - A(t)x(t)  &=& f(t), \\
 \ddt \left( E(t)\dot{x}(t) - A(t)x(t) \right) &=& f^{(1)}(t), \\
 & \dots & \\
 \left( \ddt \right)^{\mu} \left( E(t)\dot{x}(t) - A(t)x(t) \right) &=& f^{(\mu)}(t),
\eens
%
one can extract the so-called \emph{strangeness-free formulation}
%
\be\label{s-free form}
 \m{\hE_1(t) \\ 0 \\ 0} \dot{x}(t) = \m{\hA_1(t) \\ \hA_2(t) \\ 0} x(t) + \m{\hf_1(t) \\ \hf_2(t) \\ \hf_3(t)},
\ee
%
which has the same solution set as the DAE \eqref{eq4}, where the matrix-valued function $\m{\hE_1 \\ \hA_2}$ has pointwise full row rank.
For the numerical determination of the strangeness index and the strangeness-free formulation \eqref{s-free form}, we refer the readers to 
\cite{KunMRW97,KunMS05}.

Now we apply the strangeness-free formulation to the DDAE \eqref{eq1.1a}, which assumed to be causal, to obtain the strangeness-free DDAE
%
\be\label{eq5}
 \m{\hE_{1}(t) \\ 0 \\ 0} \dot{x}(t) \!=\! \m{\hA_{1}(t) \\ \hA_{2}(t) \\ 0} x(t) \!+\!
 \m{\hB_{0,1}(t) \\ \hB_{0,2}(t) \\ 0} x(t-\vtau(t))
 \!+\!  \sum_{i=1}^{\mu} \m{0 \\ \hB_{i,2}(t) \\ 0} x^{(i)}(t-\vtau(t))
 \!+\! \m{\hf_{1}(t) \\ \hf_{2}(t) \\ \hf_{3}(t)}, \quad \pm{d \\ a \\ v}
\ee
%
where $\mu=\mu(E,A)$ is the strangeness index of the function pair $(E,A)$ and the matrix-valued function $\m{\hE_1 \\ \hA_2}$ has pointwise full row rank.
Sizes of the block row equations are $d$, $a$, $v$. Due to the causality of the DDAE \eqref{eq1.1a}, it follows that $\m{\hE_1 \\ \hA_2}$ is pointwise 
nonsingular.
For the DAE \eqref{eq4}, the numerical solution $x(t)$ is obtained by integrating the strangeness-free formulation \eqref{s-free form}, which is more 
advantageous, see \cite{KunM96a,KunM96c,KunM06}.
For the DDAE \eqref{eq1.1a}, integrating the strangeness-free DDAE \eqref{eq5} is not always possible. The reason is that if 
some of the matrix functions $\hB_{i,2}$, $i=1,\dots,\mu$, is not identically zero, then the underlying DDE 
is of advanced type, which is not suitable for the numerical integration, \cite{BelZ03}. In fact, until now there is still no solver for advanced DDEs. 
For the numerical solution, solvers based on the method of steps are only suitable for retarded and neutral DDAEs, see \cite{AscP95,GugH07,Hau97,HaM14}.
In these cases, the strangeness-free DDAE \eqref{eq5} takes the form
%
\be\label{eq6}
 \m{\hE_{1}(t) \\ 0 \\ 0} \dot{x}(t) \!=\! \m{\hA_{1}(t) \\ \hA_{2}(t) \\ 0} x(t) \!+\!
 \m{\hB_{0,1}(t) \\ \hB_{0,2}(t) \\ 0} x(t-\vtau(t)) \!+\! \m{\hf_{1}(t) \\ \hf_{2}(t) \\ \hf_{3}(t)}, \quad \pm{d \\ a \\ v}.
\ee
%
The integration strategy we use for causal, retarded/neutral DDAEs of the form \eqref{eq1.1a} is: first, determine the strangeness-free formulation 
\eqref{eq6}, and second, apply numerical methods to \eqref{eq6} to compute $x(t)$.\\
For the determination of the strangeness-free DDAE \eqref{eq5}, we use the derivative arrays for DDAEs as follows
%
\bens
 E(t)\dot{x}(t) - A(t)x(t)  &=& B(t)x(t-\vtau(t)) + f(t), \\
 \ddt \left( E(t)\dot{x}(t) - A(t)x(t) \right) &=& \ddt \left( B(t)x(t-\vtau(t)) + f(t) \right), \\
 & \dots & \\
 \left( \ddt \right)^{\mu} \left( E(t)\dot{x}(t) - A(t)x(t) \right) &=& \left( \ddt \right)^{\mu} \left( B(t)x(t-\vtau(t)) + f(t) \right),
\eens
%
which can be rewritten as
%
\be\label{eq7}
M z(t) = P z(t-\vtau(t)) + g,
\ee
%
where 
%
\bens
M &:=& 
     \m{-A(t)    & E(t) & & & \\ 
    -\dot{A}(t)  & \dot{E}(t)-A(t) & E(t) & & \\ 
    -\ddot{A}(t) & \ddot{E}(t)-2\dot{A}(t) & 2\dot{E}(t)-A(t) & E(t) & \\ 
     & \vdots & & \ddots & \ddots & \\ 
-A^{({\mu})}(t)  & E^{({\mu})}(t)-{\mu}A^{({\mu}-1)}(t) & \dots  & \dots & {\mu}\dot{E}(t)-A(t) & E(t)}, \\
P  &:=& \m{B(t) &          &   &         &  & 0\\ 
 \dot{B}(t)    & B(t)        &   &         &  & 0\\ 
\ddot{B}(t)    & 2\dot{B}(t)\left( t-\vtau(t) \right)^{(1)} & B(t)\left( t-\vtau(t) \right)^{(2)} &         &  & 0\\ 
  \vdots    & & \ddots & \ddots & & \vdots \\ 
B^{({\mu})}(t)   & \binom{\mu}{1} B^{({\mu}-1)}(t) \left( t-\vtau(t) \right)^{(1)} & \dots & \binom{\mu}{\mu-1}\dot{B}(t)& B(t) \left( t-\vtau(t) \right)^{(\mu-1)} & 0} , \\
z(t) &:=& \m{x(t) \\ \dot{x}(t) \\ \vdots \\ x^{(\mu+1)}(t)}, 
\quad g := \m{f(t) \\ \dot{f}(t) \\ \vdots \\ f^{(\mu+1)}(t)}.  
\eens
%
In the following, for notational convenience, we will use Matlab notation, \cite{matlab}.
The set of algebraic constraints in the strangeness-free DDAE \eqref{eq6} is selected by finding the full row rank matrix $Z_2$ such that
%
\be\label{eq8}
 Z^T_2 M(:,(n+1):end) = 0.
\ee
%
Scaling the system \eqref{eq7} with $Z^T_2$ from the left, we obtain the equation
%
\be\label{eq9}
 Z^T_2 M(:,1:n) x(t) = Z^T_2 P z(t-\vtau(t)) + Z^T_2 g.
\ee
%
Furthermore, the DDAE \eqref{eq1.1a} is not of advanced type if and only if in \eqref{eq9} the derivatives of $x(t-\vtau(t))$ do not occur. This means 
that
%
\be\label{eq10}
 Z^T_2 P(:,(kn+1):end) = 0.
\ee
%
We consider the following spaces and matrices
%
\be\label{eq11}
\begin{array}{ccc}
 T_2 & \mbox{ basis of } & \ker(Z^T_2 E), \\
 Z_1 & \mbox{ basis of } & \range(E T_2). \\
\end{array}
\ee
%
The set of differential equations in the strangeness-free DDAE \eqref{eq6}, therefore, is
%
\[
 Z_1^T E(t)\dot{x}(t) = Z_1^T A(t)x(t) + Z_1^T B(t) x(t-\vtau(t)) + Z_1^T f(t).
\]
%
In summary, we obtain the strangeness-free DDAE
%
\be\label{eq12}
 \bsp{
 Z_1^T E(t)\dot{x}(t) &= Z_1^T A(t)x(t) + Z_1^T B(t) x(t-\vtau(t)) + Z_1^T f(t). \\
 Z^T_2 M(:,1:n) x(t)  &= Z^T_2 P(:,1:kn) x(t-\vtau(t)) + Z^T_2 g,
 }
\ee
%
where $\m{Z_1^T E(t) \\ Z^T_2 M(:,1:n)}$ is nonsingular. 

\subsection{Regularization procedure for noncausal DDAEs with single delay}\label{Sec2.2}
In order to handle noncausal DDAEs with single delay, in \cite{HaM14}, the author proposed the concept of \emph{shift index} as follows. 
For notational convenience and to be consistent to \cite{HaM14}, we will write $\tau$ instead of $\vtau$.
%
\begin{definition}\label{shift index}
Consider the IVP \eqref{eq1.1}. For each $t\in \mathbb{I}$, the minimum number $\ka = \ka(t)$ such that the \emph{shift-inflated} system
%
\be\label{eq13}
\bsp{
  E(\th{0}) \dot{x}(\th{0}) &= A(\th{0}) x(\th{0}) + B(\th{0})) x(\th{0} - \tau(\th{0})), \\  
  E(\th{1}) \dot{x}(\th{1}) &= A(\th{1}) x(\th{1}) + B(\th{1})) x(\th{1} - \tau(\th{1})), \\  
                            & \ \vdots  \\
  E(\th{\ka}) \dot{x}(\th{\ka}) &= A(\th{\ka}) x(\th{\ka}) + B(\th{\ka})) x(\th{\ka} - \tau(\th{\ka})),                            
}
\ee
%
where the sequence $\{\th{j}| j\geq 0\}$, starting with $\th{0}=t$, is determined via the equation 
%
\be\label{eq14}
 \th{j+1} - \tau(\th{j+1}) = \th{j}, \quad \mbox{for all } j\geq 0,
\ee
%
is called the \emph{shift index} of the DDAE \eqref{eq1.1a} with respect to $t$.\\
\end{definition}
%
To guarantee the existence and uniqueness of the sequence $\{\th{j}| j\geq 0\}$ in Definition \ref{shift index}, we assume that for every $s \in (t_0,t_f)$, the equation
%
\be\label{shift equation}
 t -\tau(t) = s,
\ee
%
has a unique solution on the time interval $(s,t_f)$. 

\begin{remark}
In the single delay case, if the DDAE \eqref{eq1.1a} is causal then the shift index $\ka$ is $0$ for all $t \in [t_0,t_f]$.
\end{remark}

Using \eqref{shift equation}, we can rewrite the shift-inflated system \eqref{eq13} as
%
\be\label{eq15}
\bsp{
&\m{E(\th{0})    &                            &             &     \\
                 & E(\th{1})                  &             &     \\
                 &                            & \ddots      &      \\   
                 &                            &             & E(\th{\ka}) 
                 } \! \m{\dot{x}(\th{0}) \\ \dot{x}(\th{1}) \\ \vdots \\ \dot{x}(\th{\ka})} \\
& = 
\m{A(\th{0})     &                           &             &     \\
   B(\th{1})     & A(\th{1})                 &             &     \\
                 &            \ddots         & \ddots      &      \\   
                 &                          &B(\th{\ka}) & A(\th{\ka}) 
                 } \! \m{x(\th{0}) \\ x(\th{1}) \\ \vdots \\ x(\th{\ka})}
                 \!+\! \m{B(\th{0})x(\th{0}-\tau(\th{0})) \!+\! f(\th{0}) \\ f(\th{1}) \\ \vdots \\ f(\th{\ka})}.
}
\ee
%
As shown in \cite{HaM14}, assuming that the DDAE \eqref{eq1.1a} is not of advanced type, we can extract from the DAE \eqref{eq15} a strangeness-free DDAE 
%
\be\label{eq16}
 \m{\hE_{1}(\th{0}) \\ 0} \dot{x}(\th{0}) \!=\! \m{\hA_{1}(\th{0}) \\ \hA_{2}(\th{0})} x(\th{0}) \!+\!
 \m{\hB_{1}(\th{0}) \\ \hB_{2}(\th{0})} x(\th{0}-\tau(\th{0})) \!+\! \m{\hf_{1}(\th{0}) \\ \hf_{2}(\th{0})}, \ \pm{d \\ a}
\ee
%
where $\m{\hE_{1}(\th{0}) \\ \hA_{2}(\th{0})}$ is nonsingular, $d+a=n$.\\
Analogous to the case of causal DDAEs, we build the derivative arrays \eqref{eq7} for the system \eqref{eq15}, and extract from it the 
strangeness-free DDAE \eqref{eq16}. To keep the brevity of the paper, we will not repeat the process.
It has been observed in \cite{HaM14} that the main difference between causal DDAEs and noncausal DDAEs, in the determination of the 
strangeness-free DDAEs, are:
\begin{enumerate}
 \item[i)] Size of the derivative arrays \eqref{eq7} for noncausal DDAEs is bigger than for causal DDAEs.
 \item[ii)] The set of differential equation in the strangeness-free DDAEs of noncausal DDAEs, cannot be selected from the original DDAEs, but must be 
 selected from the derivative arrays \eqref{eq7}.
\end{enumerate}

\section{Algorithms in COLDDAE}
First, we observe that the resulting system of the regularization procedure for DDAEs is the following regular, strangeness-free DDAE
%
\be\label{eq3.1}
 \m{\hE_1(t) \\ 0} \dot{x}(t) = \m{\hA_1(t) \\ \hA_2(t)} x(t) + \m{\hB_1(t) \\ \hB_2(t)} x(t-\vtau(t)) + \m{\hga_1(t) \\ \hga_2(t)}, 
\ee
%
where $\vtau(t)$ is a scalar function $\tau(t)$ in the noncausal case. We now apply numerical methods to \eqref{eq3.1} to determine $x(t)$. 
For notational convenience, in the the following we consider only the single delay case, i.e. 
$\vtau \equiv \tau$. The solver, however, can handle the multiple delays case, too.\\
Adopted from the solver RADAR5 \cite{GugH07}, we use the Radau scheme for the numerical integration, which is given by nodes
%
\be\label{eq3.2}
  0 < \de_1 < \dots < \de_s = 1, \quad s\in \hro{N}.
\ee
%
We include all the discontinuity points of $x$, $\dot{x},\dots,x^{(s)}$ into the mesh and denote a mesh by $\pi \ : \ t_0 < t_1 < \dots < t_N = t_f$.
The collocation points therefore are
%
\be\label{eq3.3}
  t_{ij} = t_i + h_i \de_j, \qquad j=1,\dots,s, 
\ee
%
where $h_i$ is the stepsize used at the $i$-th step.
For the numerical approximation of the solution, we seek for the piecewise polynomial $\mathrm{X}_{\pi}$ of degree $s$, i.e., 
$\mathrm{X}_{\pi,i}:=\mathrm{X}_{\pi}|_{[t_i,t_{i+1}]}$ are polynomials of degree $s$, which are determined by the following set of equations  
%
\be\label{eq3.4}
 \m{\hE_1(t_{ij}) \\ 0} \dot{\mathrm{X}}_{\pi}(t_{ij}) = \m{\hA_1(t_{ij}) \\ \hA_2(t_{ij})} \mathrm{X}_{\pi}(t_{ij}) + 
 \m{\hB_1(t_{ij}) \\ \hB_2(t_{ij})} \mathrm{X}_{\pi}(t_{ij}-\tau(t_{ij})) 
 + \m{\hga_1(t_{ij}) \\ \hga_2(t_{ij})}, 
%
\ee
%
for all $i=1,\dots,N$, $j=1,\dots,s$.\\
Due to the presence of $\m{\hB_1(t_{ij}) \\ \hB_2(t_{ij})} \mathrm{X}_{\pi}(t_{ij}-\tau(t_{ij}))$ in \eqref{eq3.4}, we still have to define the past function 
$\mathrm{X}_{\pi}(t_{ij}-\tau(t_{ij}))$ which is an approximation to $x(t_{ij}-\tau(t_{ij}))$. For this we choose
%
\[
\bsp{
 & \mathrm{X}_{\pi}(t_{ij}-\tau(t_{ij})) \\
 & =  
 \bc
  \phi(t_{ij}-\tau(t_{ij}))                & \mbox{ if } t_{ij}-\tau(t_{ij}) \leq 0, \\
  \mathrm{X}_{\pi,K}(t_{ij}-\tau(t_{ij}))  & \mbox{ for some $ 1 \leq K \leq N$ that satisfies } t_K < t_{ij}-\tau(t_{ij}) \leq t_{K+1}.
 \ec
} 
\]
%
The continuous output polynomial $\mathrm{X}_{\pi,K}$ at the $K$-th step is given by Lagrange interpolation of order $s$, i.e.,
%
\be\label{eq3.5}
 \mathrm{X}_{\pi,K}(t_K + \tet h_K) = \sum_{j=0}^s {\cal L}_j(\tet) \mathrm{X}_{\pi,K}(t_K + \de_j h_K),
\ee
%
where ${\cal L}_j(\tet)$ is the Lagrange polynomial of degree $s$ satisfying ${\cal L}_j(\de_K) = \de_{Kj}$ with $\de_{Kj}$ being the Kronecker delta symbol.

\begin{remark}
 As noticed in \cite{GugH01,GugH07}, one can optionally replace the continuous output polynomial $\mathrm{X}_{\pi,K}$ in \eqref{eq3.5} by another dense output polynomial given by
 %
 \[
  \mathrm{X}_{\pi,K}(t_K + \tet h_K) = \sum_{j=1}^s {\cal L}_j(\tet) \mathrm{X}_{\pi,K}(t_K + \de_j h_K).
 \]
 %
 The use of only $s$ interpolation nodes $\de_j$, $j=1,\dots,s$ instead of $s+1$ nodes $\de_j$, $j=0,\dots,s$ is beneficial in the presence of a jump in the solution at the
 point $t_K$, i.e., $\mathrm{X}_{\pi,K}(t_K) \not= \mathrm{X}_{\pi,K-1}(t_K)$.
\end{remark}

The existence and uniqueness, and the convergence results for the numerical approximation $\mathrm{X}_{\pi}$ are stated in the following theorem.

\begin{theorem}\label{Thm6.1}
Consider the IVP \eqref{eq1.1} and assume that it is uniquely solver and of either retarded or neutral type. 
For $N \in \hro{N}$ and $s \geq 1$, define the mesh $\pi$ and the collocation points $t_{ij}$, $j=1,\dots,s$ as in \eqref{eq3.3}. 
Then the following assertions hold.
\begin{compactenum}
 \item[i)] For sufficiently small mesh widths $h_0,\dots,h_{N-1}$ there exists one and only one continuous piecewise polynomial $\mathrm{X}_{\pi}$ that solves 
 the DAE sequence \eqref{eq3.4} and it is consistent at all the mesh point $t_i$.
 \item[ii)] The convergence order of the collocation method with schemes $\de_j$ as in \eqref{eq3.2} is $s$, i.e., 
 %
 \[
  \| \mathrm{X}_e(t) - \mathrm{X}_{\pi}(t) \|_{\infty} = \underset{t \in \bbI}{\sup} \| \mathrm{X}_e(t) - \mathrm{X}_{\pi}(t) \| = O(h^s),
 \]
%
 where $\mathrm{X}_e$ is the exact solution  $x \in C^{s+1}(\bbI,\C^n)$ to the IVP \eqref{eq1.1}.
\end{compactenum}
\end{theorem}
\begin{proof}
For the proof see Theorem 4, \cite{Hau97} or Theorem 4.2, \cite{GugH07}.
\end{proof}
%

\section{Using COLDDAE}

The package COLDDAE contains two smaller solvers for handling specific situations: {\tt colddae\_causal} can deal with linear, causal DDAEs with multiple time varying delays, 
while {\tt colddae\_noncausal} can handle both causal and noncausal DDAEs, but it is only applicable for systems with single delay. 
In the following we will describe the parameters inside these solvers.

\subsection{{\tt colddae\_causal} - a solver for linear, causal DDAEs with multiple delays}

We assume that the strangeness index of $(E,A)$ is not too big, otherwise large errors may occur, since we compute all derivatives by finite differences.

\subsection{Input parameters}
\begin{itemize}
\item {\tt E}\quad The matrix function $E:[t_0,t_f]\rightarrow \mathbb{R}^{m,n}$.
\item {\tt A}\quad The matrix function $A:[t_0,t_f]\rightarrow \mathbb{R}^{m,n}$.
\item {\tt B}\quad The matrix function $[B_1,\ldots,B_{k}]:[t_0,t_f]\rightarrow \mathbb{R}^{m,k n}$.
\item {\tt tau}\quad  The vector function of delays $t\mapsto[\tau_1(t),\ldots,\tau_{k}(t)]$.
\item {\tt phi}\quad The history function $\phi$, i.e. $x(t)=\phi(t)$ for $t < t_0$.
\item {\tt tspan}\quad The solution interval $[t_0,t_f]$, {\tt tspan(1)}$ = t_0$, {\tt tspan(2)}$ = t_f$ .
\item {\tt options}\quad A {\tt struct} containing the optional parameters.
\end{itemize}

\subsubsection{Optional input parameters}
Optional parameters can be passed by the input parameter {\tt options} by the command 
\begin{center}
{\tt options.}{\it field\_name} = {\it field\_value}.
\end{center}
The following fields are applicable in this solver:
\begin{itemize}
\item {\tt Iter}\quad        The number of time steps, default: {\tt 100}.
\item {\tt Step}\quad        The (constant) step size of the Runge-Kutta method, must be smaller than
          $\min_{i=1,\ldots,k}\tau_i(t)$ for all $t\in[t_0,t_f]$, default: $\frac{t_f-t_0}{100}$.
\item {\tt AbsTol}\quad       Absolute tolerance, default:  {\tt 1e-5}.
\item {\tt RelTol}\quad       Relative tolerance, default:  {\tt 1e-5}.
\item {\tt StrIdx}\quad       Lower bound for the strangeness index,  default: {\tt 0}.
\item {\tt MaxStrIdx}\quad    Upper bound for the strangeness index,  default: {\tt 3}.
\item {\tt InitVal  }\quad    Initial value, not necessarily consistent,  default: {\tt phi(tspan(1))}.
\item {\tt IsConst}\quad      A boolean, {\tt true} if $E$ and $A$ are constant (then the strangeness-free form is computed only once, i.e. the solver needs less computation time), default: {\tt false}.
\end{itemize}

\subsubsection{Output parameters}
\begin{itemize}
\item {\tt t}\quad The discretization of {\tt tspan} by {\tt Iter}$+1$ equidistant points.
\item {\tt x}\quad The numerical solution at {\tt t}.
\item {\tt info}\quad A struct with information.
\end{itemize}


\subsection{{\tt colddae\_noncausal} - a solver for linear, noncausal DDAEs with single delay}

Again, we assume that the strangeness index of $(E,A)$ is not too big. The shift index must be less or equal to 3 (because of hard coding, could be arbitrary in principle).

\subsubsection{Input parameters}
\begin{itemize}
\item {\tt E}\quad The matrix function $E:[t_0,t_f]\rightarrow \mathbb{R}^{m,n}$.
\item {\tt A}\quad The matrix function $A:[t_0,t_f]\rightarrow \mathbb{R}^{m,n}$.
\item {\tt B}\quad The matrix function $B:[t_0,t_f]\rightarrow \mathbb{R}^{m,n}$.
\item {\tt tau}\quad  The scalar delay function $t\mapsto \tau(t)$.
\item {\tt phi}\quad The history function $\phi$, i.e. $x(t)=\phi(t)$ for $t < t_0$.
\item {\tt tspan}\quad The solution interval $[t_0,t_f]$, {\tt tspan(1)}$ = t_0$, {\tt tspan(2)}$ = t_f$ .
\item {\tt options}\quad A {\tt struct} containing the optional parameters.
\end{itemize}

\subsubsection{Optional input parameters}
Optional parameters can be passed by the input parameter {\tt options} by the command 
\begin{center}
{\tt options.}{\it field\_name} = {\it field\_value}.
\end{center}
The following fields are applicable in this solver:
\begin{itemize}
\item {\tt MaxIter}\quad        Upper bound for the total number of time steps (excluding 
	rejected time steps), default: {\tt 10000}.
\item {\tt MaxReject}\quad      Upper bound for the number of rejections per time step, default: {\tt 100}.
\item{\tt MaxCorrect}\quad  Upper bound for the number of correction steps when using
         long steps (step size bigger than the lag), default: {\tt10}.
\item {\tt Step}\quad        The (constant) step size of the Runge-Kutta method, must be smaller than
          $\min_{i=1,\ldots,k}\tau_i(t)$ for all $t\in[t_0,t_f]$, default: $\frac{t_f-t_0}{100}$.
\item {\tt AbsTol}\quad       Absolute tolerance, default:  {\tt 1e-5}.
\item {\tt RelTol}\quad       Relative tolerance, default:  {\tt 1e-5}.
\item {\tt StrIdx}\quad       Lower bound for the strangeness index,  default: {\tt 0}.
\item {\tt MaxStrIdx}\quad    Upper bound for the strangeness index,  default: {\tt 3}.
\item {\tt Shift}\quad       Lower bound for the strangeness index,  default: {\tt 0}.
\item {\tt MaxShift}\quad    Upper bound for the strangeness index,  default: {\tt 3}.
\item {\tt InitVal  }\quad    Initial value, not necessarily consistent,  default: {\tt phi(tspan(1))}.
\end{itemize}

\subsubsection{Output parameters}
\begin{itemize}
\item {\tt t}\quad A discretization of {\tt tspan} with variable step size.
\item {\tt x}\quad The numerical solution at {\tt t}.
\item {\tt info}\quad A struct with information.
\end{itemize}

\section{Numerical experiments}
Ongoing!

\section{Future work}
Possible improvements include
\begin{enumerate}
\item step size control and long steps for multiple delays,
\item using exact derivatives (provided by user),
\item regularization of non-causal DDAEs with multiple delays.
\end{enumerate}

%-------------------------------------------------------------------------------
\bibliographystyle{plain}
\bibliography{HaMM14}

%\bibliography{Phi_Nov_05}
%-------------------------------------------------------------------------------
%%%%%%%%%%%%%%%%%%%%%%%%%%%%%%%%%%%%%%%%%%%%%%%%%%%%%%%
\end{document}
%%%%%%%%%%%%%%%%%%%%%%%%%%%%%%%%%%%%%%%%%%%%%%%%%%%%%%%
