% First draft - Phi 24. Nov 14
% version 0.2 - Tho 14. Dec 14
% version 0.3 - Phi 17. Dec 14
% Version 0.4 - Tho 24. Dec 14
% typos corrected - Phi 10. Jan 15

\documentclass[final,reqno]{siamltex}

\usepackage{amsfonts,epsfig}
\usepackage{amsmath,amssymb} %$#
\usepackage{graphicx}

\usepackage[square, numbers, comma, sort&compress]{natbib}  % Use the "Natbib" style for the references in the Bibliography
%\usepackage{showlabels}
%\renewcommand{\showlabelfont}{\small\slshape\color{red}}

\usepackage{algorithm}
\usepackage{algpseudocode}
\usepackage{algcompatible}

\usepackage{xcolor}
\newcommand\command[1]{\textcolor{red}{\textbf{#1}}}

% %\usepackage[notcite]{showkeys}
% \usepackage{showlabels}
% \renewcommand{\showlabelfont}{\small\slshape\color{red}}

%\usepackage{lineno}
%\linenumbers

\usepackage{paralist}
\renewenvironment{itemize}[1]{\begin{compactitem}#1}{\end{compactitem}}
\renewenvironment{enumerate}[1]{\begin{compactenum}#1}{\end{compactenum}}
\renewenvironment{description}[0]{\begin{compactdesc}}{\end{compactdesc}}

\usepackage[pdfpagelabels]{hyperref} %$#

% Common extra environments
%\newtheorem{algorithm}[theorem]{Algorithm}
\newtheorem{remark}[theorem]{Remark}
\newtheorem{example}[theorem]{Example}
\newtheorem{ass}[theorem]{Assumption}
\newtheorem{hyp}[theorem]{Hypothesis}
\newtheorem{pro}[theorem]{Procedure}

%=============================================================================
% new def-s and commands
\newcommand{\reals}{\mathbb{R}}
\newcommand{\complex}{\mathbb{C}}

\def\leq{\leqslant}
\def\rar{\rightarrow}

\def\hro{\mathbb}
\def\R{\hro{R}}
\def\C{\hro{C}}
\def\N{\hro{N}}
\def\Z{\hro{Z}}
\def\bbI{\mathbb{I}}

\def\a{\alpha}
\def\ta{\widetilde{\alpha}}
\def\b{\beta}
\def\de{\delta}
\def\De{\Delta}
\def\tet{\theta}
\def\Tet{\Theta}
\def\dch{\dot{\chi}}
\def\ch{\chi}
\def\ka{\kappa}
\def\vp{\varphi}

\def\dtau{\Delta_{\tau}}
\def\idtau{\Delta_{-\tau}}

\def\ga{\gamma}
\def\Si{\Sigma}
\def\si{\sigma}
\def\om{\omega}
\def\lb{\lambda}
\def\fr{\frac}

\def\tp{\tilde{p}}
\def\tu{\tilde{u}}

\def\frE{\mathfrak{E}}
\def\frA{\mathfrak{A}}
\def\frB{\mathfrak{B}}
\def\frC{\mathfrak{C}}
\def\frD{\mathfrak{D}}
\def\frg{\mathfrak{g}}
\def\frf{\mathfrak{f}}
\def\frX{\mathfrak{X}}
\def\frZ{\mathcal{Z}}

\def\frM{\mathcal{M}}
\def\frP{\mathcal{P}}
\def\frG{\mathfrak{G}}
\def\tfrM{\tilde{\mathcal{M}}}
\def\tfrN{\tilde{\mathcal{N}}}
\def\tfrP{\tilde{\mathcal{P}}}
\def\tfrG{\tilde{\mathcal{G}}}

\def\hfrg{\hat{\mathfrak{g}}}
\def\tfrg{\tilde{\mathfrak{g}}}
\def\hmu{\hat{\mu}}

\def\bbM{\mathbb{M}}
\def\bbN{\mathbb{N}}

\def\hr{\hat{r}}
\def\hv{\hat{v}}
\def\hm{\hat{m}}
\def\hw{\hat{w}}

\def\tP{\tilde{P}}
\def\tQ{\tilde{Q}}
\def\tH{\tilde{H}}
\def\tK{\tilde{K}}

\def\tE{\widetilde{E}}
\def\hE{\hat{E}}
\def\tA{\widetilde{A}}
\def\hA{\hat{A}}
\def\bA{\breve{A}}
\def\tB{\widetilde{B}}
\def\hB{\hat{B}}

\def\hR{\hat{R}}
\def\hS{\hat{S}}
\def\hT{\hat{T}}

\def\tR{\tilde{R}}
\def\tS{\tilde{S}}
\def\tT{\tilde{T}}

\def\tU{\tilde{U}}
\def\tk{\tilde{k}}
\def\hk{\hat{k}}
\def\tX{\tilde{X}}
\def\hX{\hat{X}}
\def\bX{\breve{X}}

\def\tm{\tilde{m}}
\def\bM{\breve{M}}
\def\hM{\hat{M}}
\def\tM{\widetilde{M}}

\def\tmu{\tilde{\mu}}

\def\bE{\breve{E}}
\def\bA{\breve{A}}
\def\bB{\breve{B}}
%\def\uE{\u{E}}
%\def\uA{\u{A}}
%\def\uB{\u{B}}

\def\bbM{\mathbb{M}}
\def\btM{\widetilde{\mathbb{M}}}

\def\hW{\widehat{W}}
\def\tW{\tilde{W}}

\def\hN{\widehat{N}}
\def\tN{\widetilde{N}}

\def\cP{{\cal P}}
\def\cQ{{\cal Q}}
\def\cU{{\cal U}}

\def\cF{{\cal F}}
\def\cG{{\cal G}}
\def\hcF{\hat{\cF}}
\def\hcG{\hat{\cG}}

\def\hcP{\hat{\cP}}
\def\hcQ{\hat{\cQ}}



\def\hS{\widehat{S}}
\def\hZ{\widehat{Z}}
\def\hH{\hat{H}}
\def\hG{\hat{G}}

\def\tx{\tilde{x}}
\def\tf{\tilde{f}}
\def\hf{\hat{f}}
\def\brf{\breve{f}}
\def\brg{\breve{g}}
\def\baf{\bar{f}}
\def\tg{\tilde{g}}
\def\hg{\hat{g}}
\def\ha{\hat{a}}
\def\hd{\hat{d}}
\def\hu{\hat{u}}

\def\cg{\mathcal{g}}
\def\ti{\times}
\def\utau{\underline{\tau}}
\def\otau{\bar{\tau}}
\def\vtau{{\tau}}
%\def\vtau{\vec{\tau}}
\def\hga{\hat{\ga}}

\renewcommand{\th}[1]{t_{(#1)}}

\def\be{\begin{equation}}
\def\ee{\end{equation}}
\newcommand{\ben}{\begin{eqnarray}}
\newcommand{\een}{\end{eqnarray}}
\newcommand{\bens}{\begin{eqnarray*}}
\newcommand{\eens}{\end{eqnarray*}}
\def\bc{\begin{cases}}
\def\ec{\end{cases}}
\newcommand{\bsq}{\begin{subequations}}
\newcommand{\esq}{\end{subequations}}

\newcommand{\m}[1]{
\begin{bmatrix}
 #1
\end{bmatrix}
}

\renewcommand{\pm}[1]{
\begin{matrix}
 #1
\end{matrix}
}

\newcommand{\doublehat}[1]{%
    \widehat{\widehat{#1\,}}
    }

\newcommand {\mpar}[1]{\marginpar{\fussy\tiny #1}} % marginal notes
\newcommand{\tcal}[1]{%
    \widetilde{\cal{#1}}
    }

    \newcommand{\hcal}[1]{%
    \widehat{\cal{#1}}
    }

    \newcommand{\bcal}[1]{%
    \breve{\cal{#1}}
    }

\newcommand{\bsp}[1]{
\begin{split}
 #1
\end{split}
}
\def\ddt{\fr{\mathrm{d}}{\mathrm{d}t}}

\newcommand{\eproof}{\space
    {\ \vbox{\hrule\hbox{\vrule height1.3ex\hskip0.8ex\vrule}\hrule}}\\[0.2cm]}
    
\def\bce{\begin{compactenum}}
\def\ece{\end{compactenum}}

\newcommand {\corank}   {\mathop{\rm corank}\nolimits}
\newcommand {\range}  {\mathop{\rm range}\nolimits}
\newcommand {\corange}  {\mathop{\rm corange}\nolimits}
\newcommand {\kernel}   {\mathop{\rm kernel}\nolimits}
\newcommand {\cokernel} {\mathop{\rm cokernel}\nolimits}
%===============================================================================
%opening
\begin{document}

\title{A numerical integration code for general linear delay differential-algebraic equations\footnotemark[1]}

\author{Phi Ha\footnotemark[2] \and Vinh Tho Ma\footnotemark[2] and Volker Mehrmann\footnotemark[2]}

\renewcommand{\thefootnote}{\fnsymbol{footnote}}

\footnotetext[1]{This work was supported by DFG Collaborative Research Centre 910,
{\it Control of self-organizing nonlinear systems: Theoretical methods and concepts of application}}
\footnotetext[2]{ Institut f\"{u}r Mathematik, MA 4-5, TU Berlin, Stra\ss e des 17. Juni 136, D-10623 Berlin,
Germany; \{ha,mavinh,mehrmann\}@math.tu-berlin.de}

\maketitle

\newcommand{\thedate}{Version 0.6 \quad \today}

\begin{center}
\thedate
\end{center}

\vskip 0.2cm

\begin{abstract}
A new numerical integration code (COLDDAE) for general linear differential-algebraic equations
(DDAEs) of retarded and neutral type is discussed. The implementation is based on the regularization technique introduced in
\cite{HaM14}. In the single delay case, the code can handle noncausal systems, i.e., systems, where the
solution at the current time does not only depend on the system at the past and current time,
but also future time points.
\end{abstract}

\begin{keywords} Delay differential-algebraic equation, differential-algebraic equation, delay differential
equations, method of steps, derivative array, classification of DDAEs.
\end{keywords}

\begin{AMS}
34A09, 34A12, 65L05, 65H10.
\end{AMS}

\pagestyle{myheadings}
\thispagestyle{plain}
\markboth{P. Ha and V. T. Ma and V. Mehrmann}{COLDDAE: a numerical integrator for linear DDAEs}

\section{Introduction}
We discuss a new numerical integration code (COLDDAE) for the numerical solution of general linear delay differential-algebraic equations (DDAEs) with variable coefficients of the following form
%
\be\label{eq1.1a}
 E(t)\dot{x}(t) = A(t)x(t) + \sum_{i=1}^k B_i(t) x(t-\tau_i(t)) + f(t), \quad t\in (t_0,t_f],
\ee
%
where $E,A, B_i \in C([t_0,t_f],\mathbb{R}^{m\times n})$,
$f\in C([t_0,t_f],\mathbb{R}^{m})$ and
the delay functions $\tau_i \in C([t_0,t_f],\mathbb{R})$ satisfy $\tau_i(t) > 0$ for all
$t\in [t_0,t_f]$.
%For later use in the index reduction procedure, we further assume that all these functions are sufficiently smooth.

Setting $\utau := \min \{\tau_i(t)| \ t\in [t_0,t_f], \ i=1,\dots,k \}$ and $\otau := \max \{\tau_i(t)| \ t\in [t_0,t_f], \ i=1,\dots,k \}$,
we assume that $\utau >0$, which is often referred in the literature \cite{BelC63,BakPT02} as the \emph{non vanishing delay} case,
and we further assume that $\otau < \infty$.

To obtain a unique solution of the DDAE \eqref{eq1.1a}, one typically needs to provide a history (or initial) function  $\phi \in C([t_0-\otau,t_0],\R^n)$ and require
%
\be\label{history}
 x(t) = \phi(t), \quad t\in [t_0-\tau,t_0).
\ee
%
In the following, we assume that such an initial function is provided and that  \eqref{eq1.1a} has a unique solution $x\in C((t_0,t_f],\mathbb{R}^n)$. The code will give an error message if this is not the case.For systems with constant delays
the theoretical analysis for problems of the form \eqref{eq1.1a} has been discussed in \cite{HaMS14,HaM14,Ha15}. These results, however,
can be directly generalized to the case of time dependent delays under some minor assumptions on the delay functions  $\tau_i$. Therefore, the solver we discuss in this paper is written to handle time dependent delays.

%The most important concepts presented in this work are the causality, the system type, the shift index and the strangeness index of the DDAE %\eqref{eq1.1a}.

Under the assumption that a unique, continuous, piecewise differentiable solution of the initial value problem \eqref{eq1.1a}, \eqref{history} exists, the implementation of the new integrator is based on the regularization procedure introduced in \cite{HaM14}, which first determines the shift-index and
the strangeness-index and then transforms the DDAE \eqref{eq1.1a} into a regular, strangeness-free formulation with the same solution set.
Using this regularization procedure, the code allows a modification of the initial values at $t_0$, if $\phi(t_0)$ is not consistent.
%, i.e., does not satisfy all (hidden) constraints,
and applies an appropriate integration scheme on the resulting regular, strangeness-free DDAE.
For the time stepping, we have implemented the three stage Radau IIA collocation method, see \cite{HaiW91}.
%
%=======================================================================================================================================
\section{A brief survey of the basic results}
%=======================================================================================================================================
%
In previous work, the numerical solution of DDAEs has only been considered for square systems, see e.g.
\cite{AscP95,BakPT02,CamL09,GugH07,Liu99,ShaG06,TiaYK11,ZhuP97,ZhuP98}.
For such systems, the solution is usually computed by the classical (Bellman) \emph{method of steps} \cite{Bel61,BelC63,BelC65}, which will be briefly summarize below. Since we assume that we have a nonvanishing delay $\utau>0$, we have $[t_0,t_f] \subset \underset{j=1,\dots,\ell+1}{\cup} [t_0+(j-1)\utau,t_0+j\utau]$
with $\ell := \lfloor \fr{t_f-t_0}{\utau} \rfloor$.
For all $t \in [t_0,t_0+\utau]$, we have $t-\tau_i(t) \leq t_0 + \utau - \utau = t_0,$ and hence $x(t-\tau_i(t)) = \phi(t-\tau_i(t))$.
 Restricted to the interval $[t_0,t_0+\utau]$ the DDAE \eqref{eq1.1a} then becomes
%
\be\label{eq2}
 E(t)\dot{x}(t) = A(t)x(t) + \sum_{i=1}^k B_i(t) \phi(t-\tau_i(t)) + f(t).
\ee
%
This system is a DAE in the variable $x_1 := x|_{[t_0,t_0+\utau]}$. The initial vector for the DAE \eqref{eq2} is $x(t_0)=\phi(0)$, and it may be consistent or not. Since we have assumed that the DAE is uniquely solvable, it will be consistent and thus the corresponding initial value problem for \eqref{eq2} with this initial value has a unique solution $x_1$.  We can proceed in the same way to compute the function $x_2 := x|_{[t_0+\utau,t_0+2\utau]}$, since $t-\tau_i(t) \leq t_0 + 2 \utau - \utau = t_0+\utau$ for all $t\in [t_0+\utau,t_0+2\utau]$. Continuing in this way, the solution $x$ of \eqref{eq1.1a} can be computed step-by-step on consecutive intervals $[t_0+(j-1)\utau,t_0+j\utau]$, $1\leq j\leq \ell$, by solving DAEs
%
\be\label{eq3}
 E(t) \dot{x}_j(t) = A(t)x_j(t) + g_j(t) \quad \mbox{for all } t \in [t_0+(j-1)\utau,t_0+j\utau],
\ee
%
where the inhomogeneity $g_j$ is obtained from the previous step.

It is obvious that the method of steps successfully handles the problem \eqref{eq1.1a} if and only if for every $j$, the corresponding initial value problem for \eqref{eq3} has a unique solution. This means that the solution $x$ at the current point $\hat t\in(t_0,t_f]$ depends only on the system \eqref{eq1.1a} at current and past time points $t \leq \hat t$, but not at future time points $t > \hat t$. We call  a a DDAE that satisfies this property \emph{causal}. Restricted to the class of causal systems, different integration strategies based on the method of steps have been successfully implemented for
linear DDAEs of the form \eqref{eq1.1a} and also for several classes of nonlinear DDAEs, see e.g. \cite{AscP95,BakPT02,GugH07,Hau97,ShaG06}.
The method of steps, however, is not feasible for noncausal systems, e.g. for the equation
%
\be\label{eq2.1}
  0 \cdot \dot{x}(t) = 0 \cdot x(t) - x(t-\tau) + f(t), \quad \mbox{for all } t\in (0,\infty)
\ee
%
the method of steps  results in a sequence of underdetermined DAEs of the form
%
\[
 0 = g_i(t), \quad \mbox{for all } t \in [t_0+(j-1)\utau,t_0+j\utau],
\]
%
despite the fact that \eqref{eq2.1} has a unique solution.
The reason for this failure is that the method of steps takes into account only the equation at the current time, which is not enough,
due to the noncausality. Therefore, to be able to apply the method of steps, a regularization procedure is necessary.

Note that for noncausal DDAEs of the form \eqref{eq1.1a}, the solvability analysis has only been studied for
the single delay case, i.e., $k=1$, and even for multiple constant delays, i.e., $\tau_i(t) \equiv \tau_i$, the problem is not entirely understood, \cite{HaM14,Ha15}.
The regularization procedure proposed in the code COLDDAE for causal and noncausal systems will be considered in the following subsections \ref{Sec2.1} and \ref{Sec2.2}, respectively.

\subsection{Regularization procedure for causal DDAEs with multiple delays}\label{Sec2.1}
Inherited from the theory of DAEs, we see that even for causal DDAEs, robust numerical integration methods require a reformulation of the original system in such a way that one can
avoid the loss in order of convergence or the drift-off effect, see e.g. \cite{BreCP96,KunM06}. Here we use the regularization procedure
associated with the \emph{strangeness index} concept, see \cite{KunM06}, which generalizes the well-known \emph{differentiation index} \cite{BreCP96} for general under- and over-determined DAEs. Loosely speaking, the strangeness index $\mu$ of the DAE
%
\be\label{eq4}
 E(t)\dot{x}(t) = A(t)x(t) + f(t),
\ee
%
is the minimum number of differentiations of the coefficients $E,A$ and the inhomogeneity $f$ such that from the \emph{derivative array} 
%(or \emph{differentiation-inflated} system)
%
\bens
 E(t)\dot{x}(t) - A(t)x(t)  &=& f(t), \\
 \ddt \left( E(t)\dot{x}(t) - A(t)x(t) \right) &=& f^{(1)}(t), \\
 & \dots & \\
 \left( \ddt \right)^{\mu} \left( E(t)\dot{x}(t) - A(t)x(t) \right) &=& f^{(\mu)}(t),
\eens
%
one can extract a so-called \emph{strangeness-free formulation}
%
\be\label{s-free form}
 \m{\hE_1(t) \\ 0 \\ 0} \dot{x}(t) = \m{\hA_1(t) \\ \hA_2(t) \\ 0} x(t) + \m{\hf_1(t) \\ \hf_2(t) \\ \hf_3(t)},
\ee
%
which has the same solution set as the DAE \eqref{eq4}, and in which where the matrix-valued function $\m{\hE^T_1 & \hA^T_2}^T$ has pointwise full row rank. We also call $\mu$ the strangeness index of the pair $(E,A)$.
For the numerical computation of the strangeness index and for extracting the strangeness-free formulation \eqref{s-free form} from the derivative array see \cite{KunMRW97,KunMS05}.

If we apply this regularization procedure to a causal DDAE of the form \eqref{eq1.1a}, then we obtain the \emph{strangeness-free DDAE}
%
\begin{equation}\label{eq5}
\begin{aligned}
 \m{\hE_{1}(t) \\ 0 \\ 0} \dot{x}(t) =&
\m{\hA_{1}(t) \\ \hA_{2}(t) \\ 0} x(t) \!+\!
\sum_{i=1}^k \m{\hB_{i,0,1}(t) \\ \hB_{i,0,2}(t) \\ 0} x(t-\tau_i(t))\\&
 \!+\!  \sum_{i=1}^k\sum_{j=1}^{\mu} \m{0 \\ \hB_{i,j,2}(t) \\ 0} x^{(j)}(t-\tau_i(t))
 \!+\! \m{\hf_{1}(t) \\ \hf_{2}(t) \\ \hf_{3}(t)}, \quad \pm{d \\ a \\ v}
\end{aligned}
\end{equation}
%
The sizes of the block row equations are $d$, $a$, $v$, respectively. Under the assumption of unique solvability and causality of the DDAE \eqref{eq1.1a}, it follows that $\m{\hE^T_1 & \hA^T_2}$ is pointwise nonsingular, and that $\hf_3$ vanishes identically, e.g. the last row can just be omitted.

Using the strangeness-free formulation, the numerical solution of \eqref{eq4} can be obtained by integrating the strangeness-free formulation \eqref{s-free form}, and in this way all the difficulties that arise in the numerical integration of high index DAEs can be avoided, see \cite{KunM96a,KunM96c,KunM06}. However, for the DDAE \eqref{eq1.1a}, integrating the strangeness-free DDAE \eqref{eq5} is not always possible. The reason is that if at least one of the matrix functions $\hB_{i,j,2}$ is not identically zero, then the underlying DDE
is of advanced type, and there may arise major difficulties in the numerical integration, see \cite{BelZ03}, and so far no numerical integration exists that can handle general advanced DDEs. Numerical methods based on the method of steps are so far only suitable for retarded and neutral DDAEs, see \cite{AscP95,GugH07,Hau97,HaM14}. In these cases, the strangeness-free DDAE \eqref{eq5} takes the form
%
\be\label{eq6}
 \m{\hE_{1}(t) \\ 0 } \dot{x}(t) \!=\! \m{\hA_{1}(t) \\ \hA_{2}(t) } x(t) \!+\!
 \sum_{i=1}^k\m{\hB_{i,1}(t) \\ \hB_{i,2}(t) } x(t-\tau_i(t)) \!+\! \m{\hf_{1}(t) \\ \hf_{2}(t)}.
\ee
%
The integration strategy that we use for causal, retarded or neutral DDAEs of the form \eqref{eq1.1a} is as follows. First, we determine the strangeness-free formulation \eqref{eq6}, and second, we apply numerical methods to \eqref{eq6} to compute $x(t)$.\\

Without loss of generality, for notational convenience, we set $k=1$, $B(t):=B_1(t)$ and  $\tau(t):=\tau_1(t)$.
For the extraction of the strangeness-free DDAE \eqref{eq6} we use the derivative array
%
\bens
 E(t)\dot{x}(t) - A(t)x(t)  &=& B(t)x(t-\vtau(t)) + f(t), \\
 \ddt \left( E(t)\dot{x}(t) - A(t)x(t) \right) &=& \ddt \left( B(t)x(t-\vtau(t)) + f(t) \right), \\
 & \dots & \\
 \left( \ddt \right)^{\mu} \left( E(t)\dot{x}(t) - A(t)x(t) \right) &=& \left( \ddt \right)^{\mu} \left( B(t)x(t-\vtau(t)) + f(t) \right),
\eens
%
which can be rewritten as
%
\be\label{eq7}
M(t) z(t) = P(t) z(t-\tau(t)) + g(t),
\ee
%
where
%
\bens
&& M(t) :=
     \m{-A(t)    & E(t) & & & \\
    -\dot{A}(t)  & \dot{E}(t)-A(t) & E(t) & & \\
    -\ddot{A}(t) & \ddot{E}(t)-2\dot{A}(t) & 2\dot{E}(t)-A(t) & E(t) & \\
     & \vdots & & \ddots & \ddots & \\
-A^{({\mu})}(t)  & E^{({\mu})}(t)-{\mu}A^{({\mu}-1)}(t) & \dots  & \dots & {\mu}\dot{E}(t)-A(t) & E(t)}, \\
&& P(t)  := \m{B(t) &          &   &         &  & 0\\
 \dot{B}(t)    & B(t) (1-\dot{\tau}(t))        &   &         &  & 0\\
\ddot{B}(t)    & 2\dot{B}(t)(1-\dot{\tau}(t)) - B(t) \ddot{\tau}(t) & B(t)(1-\dot{\tau}(t))^{2} &         &  & 0\\
  \vdots       &  \vdots & \vdots &  & & \vdots \\
   *           & *       &   *    &  & & 0
}, \\
&& z(t) := \m{x(t) \\ \dot{x}(t) \\ \vdots \\ x^{(\mu+1)}(t)},
\quad g(t) := \m{f(t) \\ \dot{f}(t) \\ \vdots \\ f^{(\mu+1)}(t)}.
\eens
%
Note that the $j$th block row of $P$ is obtained by computing the coefficients of
$\left({\ddt}\right)^{j-1}\left(B(t) x(t-\tau(t))\right)$, $j=1,\dots,\mu$.
These coefficients can be computed by using the Fa\`{a} di Bruno formula \cite{Cra05}, which gives the $m$th derivative of the
composite function $g(f(x))$ in the variable $x$ via
%
\begin{eqnarray*} %\label{Faa di Bruno}
 && \left( \ddt \right)^m \left[ g(f(x)) \right] \\
 & &\qquad = \sum \fr{m!}{b_1! \dots b_m!} g^{(k)}(f(x)) \left( \fr{f^{(1)}(x)}{1!}\right)^{b_1} \left( \fr{f^{(2)}(x)}{2!} \right)^{b_2} \dots
 \left( \fr{f^{(k)}(x)}{k!} \right)^{b_k},
\end{eqnarray*}
%
where the sum is over all possible combinations of nonnegative integers $b_1,\dots,b_k$ such that $b_1+ 2b_2 + \dots+ mb_m = m$ and
$k=b_1+b_2+\dots+b_m$. Note that, in our implementation we assume by default that $\mu \leq 3$ and the entries of $P$ are hard coded.

In the following, for notational convenience, we will use MATLAB notation, \cite{matlab} and we also omit the
argument $t$ in the coefficients of \eqref{eq7}.
The set of algebraic constraints in the strangeness-free DDAE \eqref{eq6} is selected by determining pointwise a full row rank matrix $Z_2$ such that
%
\[ %be\label{eq8}
 Z^T_2 M(:,(n+1):end) = 0.
\]%ee
%
Scaling the system \eqref{eq7} with $Z^T_2$ from the left, we obtain the equation
%
\be\label{eq9}
 Z^T_2 M(:,1:n) x(t) = Z^T_2 P z(t-\vtau(t)) + Z^T_2 g.
\ee
%
Furthermore, the DDAE \eqref{eq1.1a} is not of advanced type if and only if in \eqref{eq9} the derivatives of $x(t-\vtau(t))$ do not occur. This means
that
%
\[ %be\label{eq10}
 Z^T_2 P(:,(kn+1):end) = 0.
\] %ee
%
Consider the following spaces and matrices
%
\[ %be\label{eq11}
\begin{array}{ccc}
 T_2 & \mbox{ basis of } & \ker(Z^T_2 E), \\
 Z_1 & \mbox{ basis of } & \range(E T_2), \\
 Y_2 & \mbox{ basis of } & \range(Z^T_2 M(:,1:n)). \\
\end{array}
\] %ee
%
The set of differential equations in the strangeness-free DDAE \eqref{eq6} is then given by
%
\[
 Z_1^T E(t)\dot{x}(t) = Z_1^T A(t)x(t) + Z_1^T B(t) x(t-\vtau(t)) + Z_1^T f(t).
\]
%
In summary, we obtain the strangeness-free DDAE
%
\begin{eqnarray*} %\label{eq12}
  Z_1^T E(t)\dot{x}(t) &= & Z_1^T A(t)x(t) +  Z_1^T B(t) x(t-\vtau(t)) +  Z_1^T f(t), \\
 Y^T_2 Z^T_2 M(:,1:n) x(t)  &=& Y^T_2 Z^T_2 P(:,1:kn) x(t-\vtau(t)) + Y^T_2 Z^T_2 g,
\end{eqnarray*}
%
where either $\m{Z_1^T E(t) \\ Y^T_2 Z^T_2 M(:,1:n)}$ is square and nonsingular, or the DDAE is not or not uniquely solvable.

\subsection{Regularization procedure for noncausal DDAEs with single delay}\label{Sec2.2}
In order to handle noncausal DDAEs with a single constant delay, in \cite{HaM14} the
concept of the \emph{shift index} has been proposed, which is easily generalized to the case of time varying delays
as follows:
%
\begin{definition}\label{shift index}
Consider the problem \eqref{eq1.1a}. For each $t\in \mathbb{I}$, consider the sequence $\{\th{j}| j\geq 0\}$, starting with $\th{0}=t$,
determined via the equation
%
\be\label{eq14}
 \th{j+1} - \tau(\th{j+1}) = \th{j}, \quad \mbox{for all } j\geq 0.
\ee
%
The minimum number $\ka = \ka(t)$ such that the so-called \emph{shift-inflated} system
%
\be\label{eq13}
\bsp{
  E(\th{0}) \dot{x}(\th{0}) &= A(\th{0}) x(\th{0}) + B(\th{0}) x(\th{0} - \tau(\th{0})), \\
  E(\th{1}) \dot{x}(\th{1}) &= A(\th{1}) x(\th{1}) + B(\th{1}) x(\th{1} - \tau(\th{1})), \\
                            & \ \vdots  \\
  E(\th{\ka}) \dot{x}(\th{\ka}) &= A(\th{\ka}) x(\th{\ka}) + B(\th{\ka}) x(\th{\ka} - \tau(\th{\ka})),
}
\ee
%
uniquely determines $x(\th{0})$, is called the \emph{shift index} of the DDAE \eqref{eq1.1a} with respect to $t$.
\end{definition}

To guarantee the existence and uniqueness of the sequence $\{\th{j}| j\geq 0\}$ in Definition \ref{shift index}, we assume that for every $s \in (t_0,t_f)$ the equation
%\be\label{shift equation}
$ t -\tau(t) = s$
%\ee
has a unique solution in the time interval $(s,t_f)$.

\begin{remark}{\rm
If the DDAE \eqref{eq1.1a} is causal, then the shift index $\ka$ is $0$ for all $t \in [t_0,t_f]$.
}
\end{remark}

Using \eqref{eq14}, we can rewrite the shift-inflated system \eqref{eq13} as
%
\be\label{eq15}
\bsp{
&\m{E(\th{0})    &                            &             &     \\
                 & E(\th{1})                  &             &     \\
                 &                            & \ddots      &      \\
                 &                            &             & E(\th{\ka})
                 } \! \m{\dot{x}(\th{0}) \\ \dot{x}(\th{1}) \\ \vdots \\ \dot{x}(\th{\ka})} \\
& =
\m{A(\th{0})     &                           &             &     \\
   B(\th{1})     & A(\th{1})                 &             &     \\
                 &            \ddots         & \ddots      &      \\
                 &                          &B(\th{\ka}) & A(\th{\ka})
                 } \! \m{x(\th{0}) \\ x(\th{1}) \\ \vdots \\ x(\th{\ka})}
                 \!+\! \m{B(\th{0})x(\th{0}-\tau(\th{0})) \!+\! f(\th{0}) \\ f(\th{1}) \\ \vdots \\ f(\th{\ka})}.
}
\ee
%
As shown in \cite{HaM14}, assuming that the DDAE \eqref{eq1.1a} is uniquely solvable, not of advanced type, and that the DAE \eqref{eq15}
has a well-defined strangeness index $\mu$, we can extract from the DAE \eqref{eq15} a strangeness-free DDAE
%
\be\label{eq16}
 \m{\hE_{1}(\th{0}) \\ 0} \dot{x}(\th{0}) \!=\! \m{\hA_{1}(\th{0}) \\ \hA_{2}(\th{0})} x(\th{0}) \!+\!
 \m{\hB_{1}(\th{0}) \\ \hB_{2}(\th{0})} x(\th{0}-\tau(\th{0})) \!+\! \m{\hf_{1}(\th{0}) \\ \hf_{2}(\th{0})}, \ %\pm{d \\ a}
\ee
%
where $\m{\hE_{1}(\th{0}) \\ \hA_{2}(\th{0})}$ is square and nonsingular.

Analogous to the case of causal DDAEs, we build the derivative array \eqref{eq7} for system \eqref{eq15}, and extract from it the
strangeness-free DDAE \eqref{eq16}. We will not repeat this process here, see \cite{HaM14}.

\begin{remark}{\rm
It  has been observed in \cite{HaM14}, that the main differences between regularizing causal and noncausal DDAEs are
that the size of the derivative arrays \eqref{eq7} for noncausal DDAEs is bigger than for causal DDAEs and, furthermore, for noncausal DDAEs the set of differential equations in the strangeness-free formulation \eqref{eq6} cannot be selected from the original DDAE \eqref{eq1.1a}, but must be  selected from the derivative array \eqref{eq7}.
}
\end{remark}

\section{The code  COLDDAE}
In the code COLDDAE a numerical integration procedure is applied to the regular, strangeness-free DDAE
%
\be\label{eq3.1}
 \m{\hE_1(t) \\ 0} \dot{x}(t) = \m{\hA_1(t) \\ \hA_2(t)} x(t) + \sum_{i=1}^k\m{\hB_{i,1}(t) \\ \hB_{i,2}(t)} x(t-\tau_i(t)) + \m{\hga_1(t) \\ \hga_2(t)},
\ee
%
resulting from the regularization procedure for DDAEs,
where $k=1$ for noncausal DDAEs. For notational convenience  we describe the integration method for $k=1$.

Adopted from the solver RADAR5 \cite{GugH07}, we use a Radau collocation scheme in nodes
%
\be\label{eq3.2}
  0 < \de_1 < \dots < \de_s = 1, \quad s\in \hro{N}.
\ee
%
If we know all the  points of discontinuity of $\phi$, $\dot{\phi},\dots,\phi^{(s)}$, then we also know all the points of discontinuity 
of $x$, $\dot{x},\dots,x^{(s)}$ on $[t_0,t_f]$, and hence we can include them into the mesh.
However, for simplicity, the solver is restricted to the case that there is only a discontinuity at $t_0$.

Consider a mesh $\pi \ : \ t_0 < t_1 < \dots < t_N = t_f$ and collocation points
%
\be\label{eq3.3}
  t_{ij} = t_i + h_i \de_j, \qquad j=1,\dots,s,
\ee
%
where $h_i$ is the step size used at the $i$th step.
For the numerical approximation of the solution, we determine  piecewise polynomials $\mathrm{X}_{\pi}$ of degree $s$, i.e.,
$\mathrm{X}_{\pi,i}:=\mathrm{X}_{\pi}|_{[t_i,t_{i+1}]}$ are polynomials of degree $s$, which are determined by the following set of
interpolation equations
%
\be\label{eq3.4}
 \m{\hE_1(t_{ij}) \\ 0} \dot{\mathrm{X}}_{\pi}(t_{ij}) = \m{\hA_1(t_{ij}) \\ \hA_2(t_{ij})} \mathrm{X}_{\pi}(t_{ij}) +
 \m{\hB_1(t_{ij}) \\ \hB_2(t_{ij})} \mathrm{X}_{\pi}(t_{ij}-\tau(t_{ij}))
 + \m{\hga_1(t_{ij}) \\ \hga_2(t_{ij})},
\ee
%
for  $i=1,\dots,N$, $j=1,\dots,s$.

Due to the presence of the term
%
\[\m{\hB_1(t_{ij}) \\ \hB_2(t_{ij})} \mathrm{X}_{\pi}(t_{ij}-\tau(t_{ij}))
\]
%
 in \eqref{eq3.4}, we still have to provide a history function
$\mathrm{X}_{\pi}(t_{ij}-\tau(t_{ij}))$ which is an approximation to $x(t_{ij}-\tau(t_{ij}))$. We choose
%
\[
\bsp{
&  \mathrm{X}_{\pi}(t_{ij}-\tau(t_{ij})) \\
 & =
 \bc
  \phi(t_{ij}-\tau(t_{ij}))                & \mbox{if } t_{ij}-\tau(t_{ij}) < t_0, \\
  \mathrm{X}_{\pi,K}(t_{ij}-\tau(t_{ij}))  & \mbox{for $ 1 \leq K \leq N-1$ with } t_K < t_{ij}-\tau(t_{ij}) \leq t_{K+1},
 \ec
}
\]
%
where the continuous output polynomial $\mathrm{X}_{\pi,K}$ at the $K$th step is given by Lagrange interpolation polynomial of order $s$,
%
\be\label{eq3.5}
 \mathrm{X}_{\pi,K}(t_K + \tet h_K) = \sum_{j=0}^s {\cal L}_j(\tet) \mathrm{X}_{\pi,K}(t_K + \de_j h_K),
\ee
%
where ${\cal L}_j(\tet)$ is the Lagrange polynomial of degree $s$ satisfying ${\cal L}_j(\de_K) = \de_{Kj}$ with $\de_{Kj}$ being the Kronecker delta symbol.

\begin{remark}{\rm
 As noticed in \cite{GugH01,GugH07}, one can optionally replace the continuous output polynomial $\mathrm{X}_{\pi,K}$ in \eqref{eq3.5} by another dense output polynomial given by
 %
 \[
  \mathrm{X}_{\pi,K}(t_K + \tet h_K) = \sum_{j=1}^s {\cal L}_j(\tet) \mathrm{X}_{\pi,K}(t_K + \de_j h_K).
 \]
 %
 The use of only $s$ interpolation nodes $\de_j$, $j=1,\dots,s$ instead of $s+1$ nodes $\de_j$, $j=0,\dots,s$ is beneficial in the presence of a jump in the solution at the
 point $t_K$, i.e., $\mathrm{X}_{\pi,K}(t_K) \not= \mathrm{X}_{\pi,K-1}(t_K)$.
 }
\end{remark}

The existence and uniqueness, and the convergence results for the numerical approximation $\mathrm{X}_{\pi}$ are stated in the following theorem.
%
\begin{theorem}\label{Thm6.1}
Consider the DDAE \eqref{eq1.1a} with history function~\ref{history}. Assume that it has a unique solution $x$ and is of either retarded or neutral type.
For $N \in \hro{N}$ and $s \geq 1$, define the mesh $\pi$ and the collocation points $t_{ij}$, $j=1,\dots,s$ as in \eqref{eq3.3}.
Then the following assertions hold.
\begin{compactenum}
 \item[i)] For sufficiently small mesh widths $h_0,\dots,h_{N-1}$ there exists one and only one continuous piecewise polynomial $\mathrm{X}_{\pi}$ that solves
 the DAE sequence \eqref{eq3.4} and it is consistent at all the mesh point $t_i$.
 \item[ii)] The convergence order of the collocation method, with internal step-sizes  $\de_j$ as in \eqref{eq3.2}, is $s$, i.e.,
 %
 \[
  \| \mathrm{X}_e(t) - \mathrm{X}_{\pi}(t) \|_{\infty} = \underset{t \in \bbI}{\sup} \| \mathrm{X}_e(t) - \mathrm{X}_{\pi}(t) \| = O(h^s),
 \]
%
 where $\mathrm{X}_e$ is the exact solution.
 %  $x \in C^{s+1}(\bbI,\C^n)$ to the problem \eqref{eq1.1}.
\end{compactenum}
\end{theorem}
\begin{proof}
For the proof see Theorem~4 in \cite{Hau97} or Theorem 4.2 in \cite{GugH07}.
\end{proof}
%

\section{Using COLDDAE}

The solver COLDDAE, implemented in MATLAB, can handle both causal and noncausal DDAEs of retarded and neutral type, but as already stated above, we restrict the noncausal systems to have only one delay. If the system is of advanced type, an appropriate error message is given.
We have implemented step-size control and so-called long steps, i.e.,  the step size may become bigger than the delay.
Unless the user provides the exact derivative array, i.e., $M$, $P$ and $g$ and equation \eqref{eq7}, the strangeness index of the system \eqref{eq15} can be at most three, due to hard coding in the computation of $P$.

In the following we will describe the parameters inside the solver.

\subsection{Input parameters}
\begin{itemize}
\item {\tt E}\quad The matrix function $E:[t_0,t_f]\rightarrow \mathbb{R}^{m,n}$.
\item {\tt A}\quad The matrix function $A:[t_0,t_f]\rightarrow \mathbb{R}^{m,n}$.
\item {\tt B}\quad The matrix function $B:[t_0,t_f]\rightarrow \mathbb{R}^{m,kn}$, where $k$ is the number of delays.
\item {\tt tau}\quad  The delay functions $t\mapsto [\tau_1(t),\ldots,\tau_k(t)]$.
\item {\tt phi}\quad The history function $\phi$, i.e.,  $x(t)=\phi(t)$ for $t < t_0$.
\item {\tt tspan}\quad The solution interval $[t_0,t_f]$, {\tt tspan(1)}$ = t_0$, {\tt tspan(2)}$ = t_f$ .
\item {\tt options}\quad A {\tt struct} containing the optional parameters.
\end{itemize}

\subsection{Optional input parameters}
Optional parameters can be passed by the input parameter {\tt options} by the command
\begin{center}
{\tt options.}{\it field\_name} = {\it field\_value}.
\end{center}
The following fields are available in this solver:
\begin{itemize}
\item {\tt MaxIter}\quad        Upper bound for the total number of time steps (excluding
	rejected time steps), default: {\tt 10000}.
\item {\tt MaxReject}\quad      Upper bound for the number of rejections per time step, default: {\tt 100}.
\item{\tt MaxCorrect}\quad  Upper bound for the number of correction steps when using
         long steps (step size bigger than the lag), default: {\tt10}.
\item {\tt InitStep}\quad        The initial step size of the Runge-Kutta method, default: $\frac{t_f-t_0}{100}$.
\item {\tt MinStep}\quad         A lower bound for the step size, default: $\tt 0$.
\item {\tt MaxStep}\quad      An upper bound for the step size, default: $\tt inf$.
\item {\tt AbsTol}\quad       Absolute tolerance, default:  {\tt 1e-5}.
\item {\tt RelTol}\quad       Relative tolerance, default:  {\tt 1e-5}.
\item {\tt StrIdx}\quad       Lower bound for the strangeness index,  default: {\tt 0}.
\item {\tt MaxStrIdx}\quad    Upper bound for the strangeness index,  default: {\tt 3}.
\item {\tt Shift}\quad       Lower bound for the strangeness index,  default: {\tt 0}.
\item {\tt MaxShift}\quad    Upper bound for the shift index,  default: {\tt 3}.
\item {\tt InitVal  }\quad    Initial value, not necessarily consistent,  default: $\phi(t_0)$.
\item {\tt IsConst}\quad   A boolean, {\tt true} if $E,A,B,\tau$ are constant, the regularized system is then computed only once, default: {\tt false}.
\item {\tt DArray}\quad    Struct with three fields containing $M$, $P$ (or $[P_1,\ldots,P_k]$ for $k>1$) and $g$ from equation \eqref{eq7}.  default: not set.
\end{itemize}

\subsection{Output parameters}
\begin{itemize}
\item {\tt t}\quad A discretization of {\tt tspan} with variable step size.
\item {\tt x}\quad The numerical solution at {\tt t}.
\item {\tt info}\quad A struct with information, e.g. the strangeness index, shift index, etc.
\end{itemize}

\section{Numerical experiments}
For illustration, we use COLDDAE with its default values to solve four different non-advanced DDAEs.
Example \ref{Exa1}, \ref{Exa2} are taken from the DDE test set \cite{Pau94}.
%
\begin{example}\label{Exa1} {\rm Consider the DDE with constant delay given by
%
\bsq\label{eq20}
\begin{equation}\label{eq20a}
\bsp{
\dot{x}_1(t) & = x_3(t),\\
\dot{x}_2(t) & = x_4(t),\\
\dot{x}_3(t) & = -2mx_2(t)+(1+m^2)(-1)^mx_1(t-\pi),\\
\dot{x}_4(t) & = -2mx_1(t)+(1+m^2)(-1)^mx_2(t-\pi),
}
\end{equation}
for $t>0$ and $x(t)=\phi(t)$ for $t\leq 0$ with
\begin{equation}\label{eq20b}
\bsp{
{\phi}_1(t) & = \sin(t)\cos(mt),\\
{\phi}_2(t) & = \cos(t)\sin(mt),\\
{\phi}_3(t) & = \cos(t)\cos(mt)-m\sin(t)\sin(mt),\\
{\phi}_4(t) & =m\cos(t)\cos(mt)-\sin(t)\sin(mt).
}
\end{equation}
\esq
%
The exact  solution is $x(t)=\phi(t)$ for $t>0$ and the relative error of the numerical solution of \eqref{eq20} with $m=2$ is presented in the left part of Figure~\ref{fig_Paul}.
}
\end{example}
%
\begin{example}\label{Exa2} {\rm Consider the DDAE
%
\begin{equation}\label{eq21}
\begin{aligned}
\m{
1&-1\\
0&0
}
\dot{x}(t) &=
\m{
1& 0\\
0&1
}
x(t) +
\m{
0& 0\\
-1&0
}
x(t-1), && \mbox{for all } t>0,\\
\phi(t) &=
\m{
1\\
0
},
&& \mbox{for all } t \leq 0,
\end{aligned}
\end{equation}
%
which is is reformulated from the neutral DDE $\dot{x}(t)=x(t)+\dot{x}(t-1)$ by introducing a new variable to present $x(t-1)$. The exact solution
$x(t)= \m{x_1(t) \\ x_2(t)}$ of \eqref{eq21} is given by
%
\begin{align*}
x_1(t)=
\bc
\begin{array}{ll}
e^t & \mbox{ for } 0<t\leq1, \\
(t-1)e^{t-1}+e^t & \mbox{ for } 1<t\leq2, \\
\tfrac12(t^2-2t)e^{t-2}+(t-1)e^{t-1}+e^t & \mbox{ for } 2<t\leq3,\\
\tfrac16(t^3-3t^2-3t+9)e^{t-3}+\tfrac12 (t^2-2t)e^{t-2}+(t-1)e^{t-1}+e^t & \mbox{ for } 3<t\leq4,
\end{array}
\ec
\end{align*}
%
and
$x_2(t)=x_1(t-1)$. The relative error of the numerical solution of \eqref{eq20} is presented in the right part of Figure~\ref{fig_Paul}.
%
\begin{figure}[h]
 \centering
 \includegraphics[width=\textwidth]{plot_149_214.pdf}
 % Paul_149.pdf: 0x0 pixel, 300dpi, 0.00x0.00 cm, bb=
 \caption{Relative error of the solution of \eqref{eq20} (left) and \eqref{eq21} (right).}
 \label{fig_Paul}
\end{figure}
}
\end{example}

\begin{example}{\rm
Consider the following DDAE with constant coefficients and multiple time-varying delays. This DDAE is causal and it has
strangeness index two.
%
\begin{equation}\label{eq22}
\begin{aligned}
\m{
0&1&0\\
0&0&1 \\
0&0&0
}
\m{\dot{x}_1(t) \\ \dot{x}_2(t) \\ \dot{x}_3(t)} &=
\m{x_1(t) \\ x_2(t) \\ x_3(t)} +
\m{
x_2(t-1)+x_3(\tfrac t2-1)\\
0\\
0
}
+f(t), && \mbox{for all } t>0,\\
\phi(t) &=
\m{
e^t\\
1\\
\sin(t)
},
&& \mbox{for all } t \leq 0.
\end{aligned}
\end{equation}
%
The function $f(t)$ is chosen such that the exact solution is $x(t)=\phi(t)$. The relative error of the numerical solution of \eqref{eq22} is presented in the left part of Figure~\ref{fig_222}.
}
\end{example}

\begin{example} {\rm Consider the following linear DDAE with constant coefficients and time-varying delay. This DDAE
is noncausal and it has strangeness index one and shift index one.
%
\begin{equation}\label{eq23}
\begin{aligned}
\m{
1&0\\
0&0 \\
}
\m{\dot{x}_1(t) \\ \dot{x}_2(t)} &=
\m{
0&0\\
1&0 \\
}
\m{x_1(t) \\ x_2(t)} +
\m{
0&1\\
0&1
}
\m{x_1\bigl(t-1+\frac{\sin(t)}2\bigr) \\ x_2\bigl(t-1+\frac{\sin(t)}2\bigr)}
+f(t), && \mbox{for all } t>0,\\
\phi(t) &=
\m{
\sin(t)\\
\cos(t)
},
&& \mbox{for all } t \leq 0.
\end{aligned}
\end{equation}
%
The function $f(t)$ is chosen such that the initial value problem for  \eqref{eq23} has the unique solution $x(t)=\phi(t)$.
The relative error of the numerical solution  is presented in the right part of Figure \ref{fig_222}.
%
\begin{figure}[h!]
 \centering
 \includegraphics[width=\textwidth]{plot_222_varshifted.pdf}
 % Paul_149.pdf: 0x0 pixel, 300dpi, 0.00x0.00 cm, bb=
 \caption{Relative error of the solution of \eqref{eq22} (left) and \eqref{eq23} (right).}
 \label{fig_222}
\end{figure}
}
\end{example}
%
% \section{Future work}
% Possible improvements include
% \begin{enumerate}
% \item regularization of non-causal DDAEs with multiple delays,
% \item step size control and long steps for multiple delays,
% \item using exact derivatives for the regularization (provided by user).
% \end{enumerate}

\section*{Acknowledgment} We would like to thank Benjamin Unger for useful
suggestions and comments.

%-------------------------------------------------------------------------------
\bibliographystyle{plain}
\bibliography{HaMM15}
%\bibliography{Phi_Dec_15_14}
%-------------------------------------------------------------------------------
%%%%%%%%%%%%%%%%%%%%%%%%%%%%%%%%%%%%%%%%%%%%%%%%%%%%%%%
\end{document}
%%%%%%%%%%%%%%%%%%%%%%%%%%%%%%%%%%%%%%%%%%%%%%%%%%%%%%% 